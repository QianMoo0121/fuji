\LevelThree{shell}

\LevelFour{Purpose}
This module provides \cmd{/shell} command, which executes the command line in the host shell.

\begin{danger}{This is a dangerous module}
    This module is a powerful and dangerous module, not recommended to enable it.
\end{danger}

\LevelFour{Configuration}
\begin{Configuration}
    \item[enable\_warning]{A precautionary option to prevent this module is enabled.}
    \item[security] {
        The security options for this module.

        \begin{NestedList}
            \item[only\_allow\_console]{Only allow the console to execute \cmd{/shell} command.}
            \item[allowed\_player\_names]{Only allow the specified players to execute \cmd{/shell} command.}
        \end{NestedList}
    }

\end{Configuration}

\LevelFour{Example}

\begin{example}{Create a file using placeholder}
    \cmd{/shell touch \ph{player:name}.dangerous}
\end{example}

\begin{example}{Execute a program in the host os}
    \cmd{/shell emacs}
\end{example}

\begin{example}{Backup the data of your server}
    You can use \ttt{shell module} with \ttt{command scheduler module} as a combo: define a job to execute the shell command in os to execute a program to backup the data of your server. \\
    See more: \url{https://rdiff-backup.net/}
\end{example}

\begin{example}{Possible to download a virus from Internet and execute it!}
    \cmd{/shell ...}
\end{example}
