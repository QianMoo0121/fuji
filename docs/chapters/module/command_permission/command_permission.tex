\LevelOne{command\_permission}\label{ch:command_permission}

\LevelTwo{Purpose}
This module provides the customization of \tbf{the requirement of all commands}.

\LevelTwo{Command}
\fcmd{/command-permission}

\LevelTwo{How it works?}
The vanilla minecraft use a command system named brigadier.\\
All the commands are registered, parsed and executed by brigadier. \\
In this system, all commands are build into \tbf{a tree structure}, that is to say, all commands are a direct or in-direct child of the \tbf{root command node}.

\begin{example}{What is the path of a specific command node?}
    For example, the command \cmd{/gamemode creative Steve} is composed by 3 command node:
    \begin{description}
        \item [\str{gamemode}] = \ttt{a literal whose name is "gamemode"}
        \item [\str{creative}] = \tbf{an argument whose type is gamemode, its name is \str{gamemode} and its value is \str{creative}}
        \item [\str{Steve}] = \tbf{an argument whose type is player, its name is \str{target}, and its value is \str{Steve}}
    \end{description}
    We say that the command path of \cmd{/gamemode creative Steve}, is \ttt{["gamemode", "gamemode", "target"]}.

    \begin{tips}{How to query the name of an argument}
        You can issue \cmd{/help gamemode} which will display the name of arguments.
        Or you can issue \cmd{/fuji inspect server-commands} to query the \ttt{command path} of all commands registered in the server.
    \end{tips}
\end{example}

Also, each \tbf{command node} has its \tbf{requirement}, which is a \tbf{predicate} to check if the \tbf{command source} can use the command node.

\begin{tips}{Query the command path of a command.}
    \cmd{/lp group default permission set fuji.permission\ldots}\\
    or \cmd{/command-permission}\\
    or \cmd{/fuji inspect server-commands}
\end{tips}

\begin{warn}{Understand the parent permission node and its children node}
    It's recommend to modify the \ttt{apply-sponge-implicit-wildcard} option to \ttt{false} in \ttt{luckperms.conf}, to control the \ttt{permission node} more fine-grained.
\end{warn}

\LevelTwo{Example}
\begin{example}{Allow everyone to use \cmd{/gamemode} command}
    \cmd{/lp group default permission set fuji.permission.gamemode true}

    \begin{tips}{Allow the client-side to use gamemode switcher menu}
        After you assign the \ttt{/gamemode} command permission for players, the client-side also requires to install a mod to bypass the client-side permission checking: \url{https://modrinth.com/mod/switcher}
    \end{tips}
\end{example}

\begin{example}{Allow everyone to use \cmd{/gamemode} command except the player Alice}
    \cmd{/lp group default permission set fuji.permission.gamemode true}\\
    \cmd{/lp user Alice permission set fuji.permission.gamemode false}
\end{example}

\begin{example}{Only allow everyone to use \cmd{/gamemode spectator}}
    It's impossible to assign a single gamemode, since the command path of \cmd{/gamemode creative} and \cmd{/gamemode spectator} are both \str{gamemode.gamemode}.

    Notice that the first \str{gamemode} in the command path, means the literal argument \str{gamemode}.\\
    The second \str{gamemode} in the command path, means an argument, whose type if \ttt{gamemode}.
    This gamemode argument contains all the 4 gamemodes: adventure, creative, spectator and survival.
    That's the real reason why we can't assign a single gamemode for the command \cmd{/gamemode}.\\\\
    If you really want to assign only 1 single gamemode for everyone, you can use~\nameref{ch:command_bundle} to create a new command, which only switch the gamemode of player into spectator.

\end{example}

\begin{example}{More examples}
    See~\nameref{ch:permission}
\end{example}
